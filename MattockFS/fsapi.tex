\chapter{File-system structure as API}
\section{Base file-system structure} 
\begin{itemize}
  \item \emph{data.all} -> data/<carvpath>.all 
  \item \emph{data/} 
  \begin{itemize}
    \item \emph{<carvpath-entity>.<ext>} 
    \item \emph{<carvpath-entity>/} 
    \begin{itemize}
      \item \emph{<carvpath-entity>.<ext>} $\Longrightarrow$ ../<carvpath-entity>.<ext>
      \item \emph{<carvpath-entity>} $\Longrightarrow$ ../<carvpath-entity>
    \end{itemize}
  \end{itemize}
  \item \emph{module/}
  \begin{itemize}
    \item \emph{<module-name>.register} $\Longrightarrow$ ../instance/<instance-handle>
  \end{itemize}
  \item \emph{instance/}
  \begin{itemize}
    \item \emph{<instance-handle>/}
    \begin{itemize}
      \item \emph{<sort-pollicy>.accept} $\Longrightarrow$ ../job/<job-handle>
    \end{itemize}
  \end{itemize}
  \item \emph{job}
  \begin{itemize}
    \item \emph{<job-handle>/}
    \begin{itemize}
      \item \emph{data.all} $\Longrightarrow$ ../../data/<carvpath-entity>.<ext>
      \item \emph{data/}
      \begin{itemize}
        \item \emph{<carvpath-entity>.<ext>} $\Longrightarrow$ ../../../data/<carvpath-entity>.<ext>
      \end{itemize}
      \item \emph{newdata/}
      \begin{itemize}
        \item \emph{0+<size>.<ext>}
      \end{itemize}
    \end{itemize}
  \end{itemize}
\end{itemize}
\section{data.all}
This symbolic link is meant for human use only. It is a symbolic link to a carvpath that represents the current whole repository.
If for example at a given point in time the total repository size is 3298534883328 bytes in size, than \emph{data.all} will be a symbolic link to \emph{data/0+3298534883328.all}. There is no specific meeta-data asociated with this directory entry.
\section{\emph{data}/}
The data directory is the core of the MattockFS interface. The directory itself has its access mask set to \emph{x} only (0x111). This means that no directory listing is allowed, nor are file or directory creation actions. Files that fall within the size range of the repository don'n need to be created though, they just are, and can be accessed.
\subsection{data/<carvpath-entity>.<ext>}
Any disignation of a carvpath that is valid within the bounds of the repository can be used to get a read-only pseudo file that can be used by any forensic or non-forensic tool that expect to work on a regular file. These CarvPath files have a set of read only extended attributes to be used either by human operators, the Mattock framework or a possible future WUI. The file may be designated with any file extension as to accomodate processing tools. Some tools for processing a specific type of file won't run unless the file name has an expected file extension. For this reason, any file-extension is treated the same.
\subsubsection{data/<carvpath-entity>.<ext>::path\_state}
In order to minimize the number of OS context switches needed in framework operations, the whole set of file-system level carvpath meta-data, not related to throtling, is made accessable through a single file-system operation as a single read-only extended attribute named '\emph{path\_state}. The value of this extended attribute is a semicolon seperated list of fields with the following info:
\begin{itemize}
\item \emph{state} : The value of this field can be :
\begin{itemize}
\item "non" : This CarvPath is not currently known to the batches sub-system. 
\item "initializing" : This carvpath is currently being initialized by its creator module.
\item "anycast" : This carvpath is currently is an anycast queue, waiting for a module to process it.
\item "pending" : This carvpath has been accepted by a module but is not yet finished being processed.
\item "migrating": A migration attempt to an other host is currently taking place.
\end{itemize}
\item \emph{module} : If the state is not \emph{non}, the module that this carvpath is currently bound to.
\item \emph{mime\_type} : If the state is not \emph{non}, the mime-type of the data designated by this carvpath.
\item \emph{prefered\_extension} : If the state is not \emph{non}, the file-extension of the data designated by this carvpath.
\item \emph{hash} : If the state is not \emph{non} and if opportunistic hasing has completed, the BLAKE2 hash of the carvpath entity.
\item \emph{hash\_offset} : If the state is not \emph{non}, and if opportunistic hasing has not yet completed, the offset of the first file-data yet to be included in hash calculation.
\end{itemize}
\subsection{data/<carvpath-entity>/}
Without an extension, the CarvPath annotation in the \emph{data} directory refers to a sub directory with special purpose entries. Like the \emph{data} directory, this directory has mode \emph{0x111} and thus can't be listed. The directory provides the possibility to work with nested carvpath entities by means of flatening symbolic links, thus allowing CarvPath aware forensic tools to produce new valid CarvPath entities by designating a relative path and dereferencing the sysmbolic link.
\subsection{data/<carvpath-entity/<carvpath-entity> and data/<carvpath-entity/<carvpath-entity>.<extension>}
If an entry itself is a valid CarvPath, with or without an extension, than the entry is represented by a symbolic link back into the \emph{data} directory, carying the exact same extension. The file-system will flatten the carvpath into representing the proper data entity. For example: \emph{data/3145728+786432/1048576+65536.gif} will be a symlink to \emph{../4194304+65536.gif}.
\section{module/}
The \emph{module} directory is meant as an API into the MatockFS anycast functionality. It allows a module to register as an instance of a given module by means of a simple \emph{chdir} command. Like the \emph{data} directory, the \emph{module} directory has its access mask set to \emph{x} only (0x111). This means that no directory listing is allowed, nor are file or directory creation actions.  
\subsection{module/::throttle\_state}
FIXME: fadv\_normal,willneed,dontneed,io-access-size,queue\_size,queue\_volume
\subsection{module/<module-name>.register}
If a process wants to register as an instance of a particular module, it should do a chdir to the directory designated by this symbolic link. On each readlink invocation a new instance representation is generated within MattockFS, so the link should only be dereferenced once by every module instance. The link generated refers to a unique per instance directory where the module instance can start accepting jobs and cooperating with the rest of the Mattock modules.
\subsection{instance/}
Like the \emph{data} and \emph{module} directory, the \emph{instance} directory  has its access mask set to \emph{x} only (0x111). This means that no directory listing is allowed, nor are file or directory creation actions. The directory is meant as holding point for instance handle pseudo directories.
\subsection{instance/<instance-handle>/}
Each module instance has its own instance-handle directory. The  access mask for this directory is set to \emph{x} only (0x111).  This means that no directory listing is allowed, nor are file or directory creation actions. If the module queue is non empty, the directory contains a symbolic link entry \emph{<sort-pollicy>.accept} for any valid sort-pollicy. If the module is the \emph{kickstart} module, than instead of a possible \emph{<sort-pollicy>.accept} symlink, there is a \emph{kickstart} sym-link.
\subsection{instance/<instance-handle>/<sort-pollicy>.accept}
\subsection{instance/<instance-handle>/kickstart}
\section{job/}
\subsection{job/<job-handle>
\subsection{job/<job-handle>/data.all}
\subsection{job/<job-handle>/data/}
\subsection{job/<job-handle>/data/<carvpath-entity>.<extension>}
\subsubsection{job/<job-handle>/data/<carvpath-entity>.<ext>::submit}
\subsection{job/<job-handle>/newdata/}
\subsection{job/<job-handle>/newdata/0+<size>.<extension>
\subsubsection{job/<job-handle>/newdata/0+<size>.<extension>::immutable\_path}
\section{Framework use-cases}
\subsection{Register as module}
\subsection{Kickstarting an image file}
\subsection{Closing the toolchain as data-store module}
\subsubsection{Closing the toolchain for evidence data}
\subsubsection{Closing the toolchain for evidence meta-data}
\subsubsection{Closing the toolchain for evidence provenance meta-data}
\subsection{Processing evidence data}
\subsubsection{Processing evidence data, side-effects only}
\subsubsection{Processing evidence-data and extracting meta-data}
\subsubsection{Processing evidence-data and deriving a child from a sub-carvpath}
\subsubsection{Processing evidence-data and deriving a child from extracted data}
\subsection{Load-balancing}
\subsubsection{Monitoring page-cache pressure and disk-cache miss rates} 
\subsubsection{Migrating batches to other nodes}
\subsubsection{Accepting batches from other nodes}
